\documentclass[12pt,letterpaper]{article}

\usepackage[table]{xcolor}
\usepackage[margin=1in]{geometry}
\usepackage{fancyhdr}
\usepackage{amsmath}
\usepackage{tikz}
\usepackage{tabularx}
\usepackage{changepage}
\DeclareUnicodeCharacter{2217}{*}
\pagestyle{fancy}
\rhead{Homework 2}
\lhead{\textbf{Phien Nguyen}}

\begin{document}
\section{Probability Spaces}
Let $\Omega$ = {1, 2, 3, 4, 5, 6}, A = {$\theta$, $\Omega$, {1, 2}, {3, 4, 5, 6}} and P : A → R be defined as follows:
P ($\theta$) = 0, P ($\Omega$) = 1, P ({1, 2}) = 0.1, P ({3, 4, 5, 6}) = 0.9. Is ($\Omega$, A, P ) a probability space or not?
Explain your answer (simply answering Yes/No will give you no credit.)
\\
\\
A probability distribution is subject to the following constraints:
\begin{itemize}
    \item P($\Omega$) = 1
    \item P(A) $\ge 0$ for each A $\in \alpha$
    \item $if A, B \in A and A \cap B = \theta, then P(A \cup B) = P(A) + P(B)$
\end{itemize}
\leavevmode
\\
\textbf{Yes this is probability space given proof:}
\begin{itemize}
    \item One condition has been fulfilled as stated in the parameters, $P(\Omega) = 1$
    \item Given that $P(\Omega) = 1$, we can prove this given that $A = (1, 2)$ and $B = (3, 4, 5, 6)$. The probability of both and A and B, $P(A) = 0.1$ and $P(B) = 0.9$ where $P(A) + P(B) = 1$
    \item Given that each probability of each number where, $P({1}) = P({2}) = P({3}) = P({4}) = P({5}) = P({6}) = \frac{1}{6}$, therefore satisfying that $P(A) \ge 0$
\end{itemize}
\leavevmode\newline
\\
\section{Roulette Playing}
Consider the game of roulette, where there are 37 pockets on the wheel (numbered from 0 to 36) and the wheel is fair, i.e., all pockets have the same probability. The 0 pocket is green, even numbers are black, and odd numbers are red (this is not true in reality, but it makes things simpler.) Compute the following probabilities:
\\
\\
1. Consider that Black is only $18/37$ given that 1 is green and the other half. Black represents half the numbers between 0 - 36. Next we are given out of the blacks pockets, we need to find the probability of that the number is a multiple of 3 which would give us a set of {6, 12, 18, 24, 30, 36} given it a probability of 6/37. To find the probability of the outcome being divisible by 3 given that it is black. Therefore P(A$\vert$B) = $\frac{\frac{6}{37}}{\frac{18}{37}} = \frac{6}{18} = \frac{1}{3}$
\\
2. Consider that there are 4 numbers that are divisible by 3 as listed {3, 6, 9, 12} and 3 that are divisible by 4 as listed {4, 8, 12}. $P(/_{4}\cup/_{3}\vert R^{1-12}) = \frac{\frac{6}{37}}{\frac{12}{37}} = 0.5$
\\
\section{Bayes Rule}
Let us now apply this rule to the scenario where a robot moves along a corridor where there is a door and is equipped with a sensor to determine if the robot is in front of the door or not (for 1 example, a computer vision algorithm processing images coming from a camera on top of the robot.) To simplify things, assume that there are only two locations for the robot, i.e., the robot is either in front of the door or not. These two possibilities are modeled by these two events
\\
\\
$P(B) = P(B\vert A)\cdot P(A) + P(B\vert\bar{A}) \cdot P(\bar{A}) $
\\
$P(B) = P(B\vert A)\cdot P(A) + P(B\vert\bar{A}) \cdot P(\bar{A}) $
\\
$P(B) = 0.9 \cdot P(A) + 0.2 \cdot (1 - P({A})) $
\\
$P(B) = 0.9 \cdot .5 + 0.2 \cdot .5 $
\\
$P(B) = 0.45 + 0.1$
\\
$P(B) = 0.55$

\end{document}
